% !TeX spellcheck = en_GB
% !TeX program = lualatex
%
% v 2.3  Feb 2019   Volker RW Schaa
%		# changes in the collaboration therefore updated file "jacow-collaboration.tex"
%		# all References with DOIs have their period/full stop before the DOI (after pp. or year)
%		# in the author/affiliation block all ZIP codes in square brackets removed as it was not %         understood as optional parameter and ZIP codes had bin put in brackets
%       # References to the current IPAC are changed to "IPAC'19, Melbourne, Australia"
%       # font for ‘url’ style changed to ‘newtxtt’ as it is easier to distinguish "O" and "0"
%
\documentclass[a4paper,
               %boxit,        % check whether paper is inside correct margins
               %titlepage,    % separate title page
               %refpage       % separate references
               %biblatex,     % biblatex is used
               keeplastbox,   % flushend option: not to un-indent last line in References
               %nospread,     % flushend option: do not fill with whitespace to balance columns
               %hyphens,      % allow \url to hyphenate at "-" (hyphens)
               %xetex,        % use XeLaTeX to process the file
               %luatex,       % use LuaLaTeX to process the file
               ]{jacow}
%
% ONLY FOR \footnote in table/tabular
%
\usepackage{pdfpages,multirow,ragged2e} %
\usepackage{tablefootnote}
%
% CHANGE SEQUENCE OF GRAPHICS EXTENSION TO BE EMBEDDED
% ----------------------------------------------------
% test for XeTeX where the sequence is by default eps-> pdf, jpg, png, pdf, ...
%    and the JACoW template provides JACpic2v3.eps and JACpic2v3.jpg which
%    might generates errors, therefore PNG and JPG first
%
\makeatletter%
	\ifboolexpr{bool{xetex}}
	 {\renewcommand{\Gin@extensions}{.pdf,%
	                    .png,.jpg,.bmp,.pict,.tif,.psd,.mac,.sga,.tga,.gif,%
	                    .eps,.ps,%
	                    }}{}
\makeatother
\newcommand{\ts}{\textsuperscript}
% CHECK FOR XeTeX/LuaTeX BEFORE DEFINING AN INPUT ENCODING
% --------------------------------------------------------
%   utf8  is default for XeTeX/LuaTeX
%   utf8  in LaTeX only realises a small portion of codes
%
\ifboolexpr{bool{xetex} or bool{luatex}} % test for XeTeX/LuaTeX
 {}                                      % input encoding is utf8 by default
 {\usepackage[utf8]{inputenc}}           % switch to utf8

\usepackage[USenglish]{babel}
\usepackage{subcaption}

%
% if BibLaTeX is used
%
\ifboolexpr{bool{jacowbiblatex}}%
 {%
  \addbibresource{jacow-test.bib}
  \addbibresource{biblatex-examples.bib}
 }{}
\listfiles

%%
%%   Lengths for the spaces in the title
%%   \setlength\titleblockstartskip{..}  %before title, default 3pt
%%   \setlength\titleblockmiddleskip{..} %between title + author, default 1em
%%   \setlength\titleblockendskip{..}    %afterauthor, default 1em

\begin{document}

\title{Medida da tabela de feedforward do EPU50}

\author{Envolvidos: FAC - LNLS \\ Segunda-feira, 23 de janeiro de 2023}
\maketitle
%
\section{Resumo}
Continuação dos estudos de caracterizações do efeito do EPU50 no feixe do anel de armazenamento. Neste estudo foi medida a tabela de feedforward das corretoras do wiggler para minimizar as distorções de órbita causadas pelas primeiras e segundas integrais deste ID. 

\section{Procedimento de medida}

\begin{enumerate}
    \item Zerar correntes das corretoras de feedforward
    \item Mover o gap do EPU50 para o máximo ($\SI{300}{\milli\meter}$)
    \item Corrigir a órbita com o SOFB e registrar os kicks iniciais $\left(\vec{\theta}_{x, 0}, \vec{\theta}_{y, 0}\right)$
    \item Mover o gap do ondulador para o valor de interesse, com isso a órbita será distorcida
    \item Corrigir a órbita com o SOFB. Esta etapa é necessária para medir a matriz resposta próximo da órbita ideal
    \item Medir a matriz resposta de órbita das quatro corretoras do EPU50
    \item Desfazer a correção feita com o SOFB, reaplicando os kicks iniciais que foram registrados $\left(\vec{\theta}_{x, 0}, \vec{\theta}_{y, 0}\right)$
    \item Com a matriz resposta das corretoras feedforward, calcular e aplicar iterativamente os kicks necessários para corrigir a distorção gerada pelo EPU na fase de interesse. 
    \item Após convergência da correção de órbita com as corretoras feedforward, registrar kicks $\left(\theta_{x, \mathrm{CH1}},\theta_{x, \mathrm{CH2}}\right)$ para o gap avaliado
\end{enumerate}

Seguindo o procedimento descrito, cria-se uma tabela de kicks $\left(\theta_{x, \mathrm{CH1}},\theta_{x, \mathrm{CH2}}\right)$ em função do gap do ondulador.

As etapas 2 e 3 podem ser realizadas apenas no processo para medir o primeiro valor de gap. Uma vez registrados os kicks iniciais uma vez, eles podem ser utilizados como referência para medidas em outros gaps. 

Outra observação importante é que com este procedimento, estamos convencionando que para o gap máximo do EPU de $\SI{300}{\milli\meter}$, as correntes das corretoras serão zero.

\section{Resultados}
Fizemos a medida com $\SI{5}{\milli\ampere}$ armazenados na máquina e com a configuração que corrige ótica e acoplamento para gap mínimo de $\SI{49.73}{\milli\meter}$. Seguimos o procedimento descrito anteriormente para diversos valores de gap, obtendo assim a tabela~\ref{tab:ffwd_table}.

\begin{table}
\caption{Tabela de feedforward para o EPU50. As corretoras de feedforward são SI-10SB:PS-CH-1 e SI-10SB:PS-CV-1 (upstream) e SI-10SB:PS-CH-2 e SI-10SB:PS-CV-2 (downstream).}
\centering
\begin{tabular}{ccccc}
\hline
\hline
gap {[}mm{]} & CH-1 {[}A{]} & CH-2 {[}A{]} & CV-1 {[}A{]} & CV-2 {[}A{]} \\
\hline\hline
36.0  & $+$0.8508   & $-$0.1625   & $+$0.2013   & $-$1.4226 \\ \hline
41.0  & $+$0.6462   & $-$0.1209   & $+$0.3925   & $-$1.5544 \\ \hline
47.0  & $+$0.4696   & $-$0.0991   & $+$0.5354   & $-$1.5957 \\ \hline
58.0  & $+$0.1902   & $-$0.0345   & $+$0.5643   & $-$1.6032 \\ \hline
81.0  & $-$0.0704   & $+$0.0093   & $+$0.5419   & $-$1.2614 \\ \hline
92.0  & $-$0.1403   & $+$0.0434   & $+$0.3727   & $-$1.1094 \\ \hline
150.0 & $-$0.0871   & $+$0.0144   & $+$0.1493   & $-$0.5099 \\ \hline
300.0 & $+$0.0000   & $+$0.0000   & $+$0.0000   & $+$0.0000 \\
\hline
\hline
\end{tabular}
\label{tab:ffwd_table}
\end{table}

Graficamente, os dados da Tabela~\ref{tab:ffwd_table} estão representados na Figuras ~\ref{fig:ffwd_table_h} e ~\ref{fig:ffwd_table_v}.
\begin{figure}
    \centering
    \includegraphics[width=0.47\textwidth]{idff_horizontal.png}
    \caption{Corrente das corretoras horizontais do EPU50 necessárias para corrigir a distorção de órbita em função do gap. Os dados do gráfico estão na Tabela}
    \label{fig:ffwd_table_h}
\end{figure}
\begin{figure}
    \centering
    \includegraphics[width=0.47\textwidth]{idff_vertical.png}
    \caption{Corrente das corretoras verticias do EPU50 necessárias para corrigir a distorção de órbita em função do gap. Os dados do gráfico estão na Tabela}
    \label{fig:ffwd_table_v}
\end{figure}


\end{document}
